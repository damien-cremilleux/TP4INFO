\documentclass[a4paper]{article}

\usepackage{ucs}
\usepackage[utf8x]{inputenc}
\usepackage[T1]{fontenc}

\usepackage[french]{babel}

\usepackage{graphicx}
\usepackage{subfigure}

\usepackage{listings}
\usepackage{listingsutf8}
\usepackage{mathtools}
%% \usepackage{pst-tree}
%% \usepackage{pdftricks}
%% \begin{psinputs}
%%    \usepackage{pstricks}
%%    \usepackage{multido}
%% \end{psinputs}
\usepackage{qtree}
 
\lstset{
mathescape=true,
breaklines=true
}


%%%%%%%%%%%%%%%%%%%%%%%%%%%%%%%%%%%%%%%%%%%%%%%%%%%%%%%%%%%%

\author{
  Damien \textsc{Crémilleux}, Tom \textsc{Demulier--Chevret} \\ \\
  INSA de Rennes \\
  4INFO, LSR-B
}

\title{TP2 Compilation \\ Analyse syntaxique descendante}
\date{}

\begin{document}
\maketitle

\section{Compte-rendu}

\subsection{Grammaire}

\subsubsection{Nouvelle grammaire utilisée}

La grammaire 1.2 a été réécrite sous une forme équivalente qui ne comporte plus d'expression  régulière.

\begin{lstlisting}[frame=single]
 1. <Expr>          --> <Termb> <SuiteExpr>
 2. <SuiteExpr>     --> ``ou'' <Expr> | $\epsilon$
 3. <Termb>         --> <Facteurb> <SuiteTermb>
 4. <SuiteTermb>    --> ``et'' <Termb> | $\epsilon$
 5. <Facteurb>      --> <Relation> | ``(`` <Expr> ``)'' |
                      ``si'' <Expr> ``alors'' <Expr> ``sinon'' <Expr> ``fsi''
 6. <Relation>      --> <Ident> <Op> <Ident>
 7. <Op>            --> ``='' | ``<>'' | ``<'' |
                      ``>'' | ``>='' | '<=''
 8. <Ident>         --> <Lettre> <SuiteIdentEt>
 9. <SuiteIndentEt> --> <SuiteIndent> <SuiteIndentEt> | $\epsilon$
 10.<SuiteIdent>    --> <Lettre> | <Chiffre>
\end{lstlisting}

\subsubsection{Preuve de la propriété LL(1)}

Les règles 1, 3, 6 et 8 de la grammaire de comporte pas de ou, il n'y a donc pas de vérification à faire les concernant.

Ce n'est pas le cas des règles 2, 4, 5, 7. Ident est considéré comme un terminal, il n'y a donc pas de vérification à effectuer pour les règles 9 et 10.

\paragraph{Vérification de la règle 2}
~
\begin{lstlisting}
premier(``ou'' <Expr>) $\cap$ premier($\epsilon$) = {``ou''} $\cap$ {$\epsilon$} = $\emptyset$

null(<SuiteExpr>) = null (``ou'' <Expr>) $\vee$ null($\epsilon$) = vrai
premier(<SuiteExpr>) = premier(``ou'' <Expr>) $\cup$ premier($\epsilon$) = {``ou''}
suivant(<SuiteExpr>) =  suivant(<Expr>)
suivant(<Expr>) = suivant(<SuiteExpr>)
\end{lstlisting}

\begin{center}
    \begin{tabular}{ | l | l | l | l |}
      \hline
      suiv\_Expr & \{\} & $\emptyset$ & $\emptyset$ \\ \hline
      suiv\_SuiteExpr & \{\} & $\emptyset$ & $\emptyset$ \\
      \hline
    \end{tabular}
\end{center}

Par Kleene, premier(<SuiteExpr>) $\cap$ suivant(<SuiteExpr>) = $\emptyset$

\begin{lstlisting}
null( ``ou'' <Expr>) = faux
null($\epsilon$) = vrai
\end{lstlisting}

La règle 2 vérifie donc bien les propriétés d'une grammaire LL(1).

\paragraph{Vérification de la règle 4}
~
\begin{lstlisting}
premier(``et'' <Termb>) $\cap$ premier($\epsilon$) = {``et''} $\cap$ {$\epsilon$} = $\emptyset$

null(<SuiteTermb>) = null (``et'' <Termb>) $\vee$ null($\epsilon$) = vrai
premier(<SuiteTermb) = premier(``et'' <Termb>) $\cup$ premier($\epsilon$) = {``et''}
suivant(<SuiteTermb>) =  suivant(<Termb>)
suivant(<Termb>) = suivant(<SuiteTermb>)
\end{lstlisting}

\begin{center}
    \begin{tabular}{ | l | l | l | l |}
      \hline
      suiv\_Termb & \{\} & $\emptyset$ & $\emptyset$ \\ \hline
      suiv\_SuiteTermb & \{\} & $\emptyset$ & $\emptyset$ \\
      \hline
    \end{tabular}
\end{center}

Par Kleene, premier(<SuiteTermb>) $\cap$ suivant(<SuiteTermb>) = $\emptyset$

\begin{lstlisting}
null( ``et'' <Termb>) = faux
null($\epsilon$) = vrai
\end{lstlisting}

La règle 4 vérifie donc bien les propriétés d'une grammaire LL(1).

\paragraph{Vérification de la règle 5}
~
\begin{lstlisting}
premier(<Relation>) = premier(<Ident>) = {``ident''}
premier(``(`` <Expr> ``)'') = {``(``}
premier(``si'' <Expr> ``alors'' <Expr> ``sinon'' <Expr> ``fsi'') = {``si''}
{``ident''} $\cap$ {``(``} $\cap$ {``si''} = $\emptyset$

null(<Facteurb>) = faux
\end{lstlisting}
La règle 5 vérifie donc bien les propriétés d'une grammaire LL(1).

\paragraph{Vérification de la règle 7}
~
\begin{lstlisting}
premier(``='') $\cap$ premier(``<>'') $\cap$ premier(``<'') $\cap$ premier(``>'') $\cap$ premier(``<'') $\cap$ premier(``>='') $\cap$ premier(``<='') = $\emptyset$

null(<Op>) = faux
\end{lstlisting}
La règle 7 vérifie donc bien les propriétés d'une grammaire LL(1).

%% \subsubsection{Arbre concret}

%% \Tree [ .t~>~y~et~(x~=~y) [.<Termb> <Facteurb> <>] <SuiteExpr> ]


%% \Tree [ .Game~of~Throne House~Stark House~Baratheon House~Lannister House~Targaryen House~Arryn ]


%% \section{Arbre plus complexe}
%% \Tree [ .Game~of~Throne [ .House~Stark [ .Eddard~Stark [ .Jon~Snow ] ] [ .Catelyn~Stark [ .Robb~Stark ] [ .Sansa~Stark ] [ .Arya~Stark ] [ .Bran~Stark ] [ .Rickon~Stark ] ] ] [ .House~Baratheon [ .King~Robert~Baratheon Renly~Baratheon ][ .Queen~Cersei~Baratheon Myrcella~Baratheon Joffrey~Baratheon Tommen~Baratheon ] ][ .House~Lannister [ .Tywin~Lannister [ .Tyrion~Lannister ] [ .Jaime~Lannister ] ] ] [ .House~Targaryen [ .Daenerys~Targaryen ] [ .Viserys~Targaryen ] ] [ .House~Arryn [ .Lysa~Arryn ] [ .Jon~Arryn [ .Robin~Arryn ] ] ] ]




\subsection{Questions de compréhension du TP}

\paragraph{Question 2.1}
Les commentaires sont éliminés lors de l'analyse lexicale, ils n'ont donc pas besoin d'être inclus dans la grammaire définissant les constructeurs du langage.

\paragraph{Question 2.2}
L'implémentation réalisée repose sur la construction d'un arbre. Cette structure de donnée permet de mémoriser les éléments à compter, à l'image d'un automate à pile.

\paragraph{Question 2.3}
L'intérêt d'utiliser une grammaire LL(1) est de pouvoir effectuer une analyse descendante complètement déterministe. En effet la pré-lecture de la prochaine unité lexicale suffit à garantir l'unicité de la règle de dérivation.

\paragraph{Question 2.4}
L'intérêt d'utiliser un arbre abstrait réside dans le fait de manipuler un objet beaucoup plus léger, qui ne contient plus que la structure profonde indispensable.

\section{Sources}
Les sources OCaml sont disponibles dans le dossier Sources. Le fichier \textit{ulex.ml} contient la déclaration de type des unités lexicales, tandis que \textit{lexer.mll} décrit l'analyseur lexical.
L'analyse syntaxique descendante est réalisée dans le fichier \textit{anasynt.ml}.

\section{Tests}

\end{document}
