\documentclass[a4paper]{article}

\usepackage[utf8]{inputenc}
\usepackage[french]{babel}
\usepackage{listings}

\begin{document}
\title{DS 2012 Parallélisme}
\author{Damien Crémilleux, Lauriane Holy, Boris Labbé}
\date{\today}

\maketitle

\section{Problème 1 : Le salon de coiffure et les moniteurs de Hoare}

\subsection{Décrire les processus coiffeur et client}

\begin{lstlisting}[frame=single]
  Class Client {
    SalonCoiffure.se_faire_coiffer();
  }

  Class Coiffeur {
    while(true) {
      SalonCoiffure.client_suivant();
      SalonCoiffure.terminer_coupe();
    }
  }   
\end{lstlisting}

\subsection{Donner le texte du moniteur}
\begin{lstlisting}[frame=single, breaklines=true]
  Class SalonCoiffure implements HOARE_Monitor { //Non java - les methodes s'executent toutes en exclusion mutuelle
    
    private condition coiffeur_disponible; //declaration de la file (fauteuil) pour le coiffeur
    private condition fauteil_occupe; //declaration de la file (fauteuil) le client
    private condition porte_ouverte; //declaration de la file (ouverture de la porte) pour le client
    private condition client_sorti; //declaration de la file (attente que le client sorte) pour le client
    
    public se_faire_coiffer() {
      coiffeur_disponible.reprendre();
      fauteuil_occupe.attendre();
      porte_ouverte.attendre();
      client_sortie.attendre();
    }
    
    public client_suivant() {
      coiffeur_disponible.attendre();
    }
    
    public terminer_coupe() {
      //coiffure du client
      fauteuil_occupe.reprendre();
      porte_ouverte.reprendre();
      client_sortie.reprendre();
    }
  }
  
\end{lstlisting}

\section{Problème 2 : Simulation de feux tricolores et CSP}

\subsection{Automate élémentaire}

\begin{lstlisting}[breaklines=true, frame=single, basicstyle=\footnotesize]
  
  *[true --> [etat = vert --> delay(d1); etat = orange; feu(orange) 
              |
              etat = orange --> delay(d2); etat = rouge; feu(rouge) 
              |
              etat = rouge --> delay(d3); etat = vert; feu(vert)]
   ]
\end{lstlisting}

\subsection{Synchronisation 2 feux}

\begin{lstlisting}[breaklines=true, frame=single, basicstyle=\footnotesize]
  automate(i) : 0..1
  int go;
  [i = 0 --> etat = vert; feu(vert);
   |
   i = 1 --> etat = rouge; feu(rouge);
  ]
*[true --> [etat = vert --> delay(d1); etat = orange; feu(orange)
            |
            etat = orange --> delay(d2); etat = rouge; feu(rouge); automate((i+1)%2)!go 
            |
            etat = rouge;automate((i+1)%2)?go--> delay(d3); etat = vert; feu(vert)]
  ]
  
\end{lstlisting}

\subsection{Synchronisation des véhicules}


\begin{lstlisting}[breaklines=true, frame=single, basicstyle=\footnotesize]
  automate(i) : 0..1
  int go;
  int j;
  [i = 0 --> etat = vert; feu(vert); 
   |
   i = 1 --> etat = rouge; feu(rouge);
  ]
*[true --> [etat = vert --> delay(d1); etat = orange; feu(orange) 
            |
            etat = orange --> delay(d2); etat = rouge; feu(rouge); automate((i+1)%2)!go 
            |
            etat = rouge;automate((i+1)%2)?go--> delay(d3); etat = vert; feu(vert); (j:1..nb_voiture)voiture(j)!i] //on envoie le numero du feu qui passe au vert a toutes les voitures
 ]

 borismobile(j) : 1..nb_voiture
  int feu_attente;
  int i;
  int k;
  *[(k:0..1)automate(k)?i  --> [i = feu_attente --> passe //on regarde que le message envoye provient du bon
                                | 
                                i != feu_attente --> stop]
   ]
  
\end{lstlisting}

\end{document}
